\documentclass{article}

\usepackage[english]{babel}

\usepackage[a4paper, top=2cm, bottom=2cm, left=3cm, right=3cm, marginparwidth=1.75cm]{geometry}

\usepackage{amsmath}
\usepackage{graphicx}
\usepackage[colorlinks=true]{hyperref}

\title{ElasticBridge: Trust-less Cross-Chain Bridge for Transferring Rebase Tokens}
\author{Timothy Yalugin\\timauthx@gmail.com}

\begin{document}
\maketitle

\begin{abstract}

This paper proposes an approach for secure and robust bridging of ERC-20 compatible rebase tokens between EVM-based source chain and Substrate-based parachain deployed on Polkadot network.

\end{abstract}

\section{Introduction}

Emerging applications of blockchain technology indicate an ongoing trend for market diversification. Once the idea of sovereignty-preserving application-specific chains became feasible, it quickly displace a single \textit{chain-maximalism} dogma. We are now live in a world where \textit{multi-chain} networks like Polkadot \cite{wood2016polkadot} and Cosmos \cite{kwon2016cosmos} are gathering diverse interconnected applications around their relay-chains and hub-chains respectively, creating \textit{the Internet of Blockchains\footnote{\url{https://cosmos.network}}}.

Frameworks like Parity's Substrate\footnote{\url{https://docs.substrate.io}} and Tendermint's Cosmos SDK\footnote{\url{https://tendermint.com/sdk}} streamlined the process building and deploying of sovereign and flexible blockchains, allowing them to have different consensus mechanisms, finalities, state models, governance, etc.

\subsection{Cross-chain Bridges}

Effectively improved on scalability by shading on-chain data and parallelising transaction execution, we encountered new challenges with cross-chain interoperability. Many newborn ecosystems, all with unique features and value propositions, found themselves near of Tower of Babel\footnote{Myth} --- unable to communicate and effectively cooperate to provide meaningful services together.

The natural reaction was a development of cross-chain bridges --- protocols that allow two (or more) exogenous chains to share state by the means of transaction passing. We have seen various implementations of these protocols ranging from trusted federations to truly trust-less decentralized Relayed SPV solutions like the IBC Protocol \cite{goes2020ibc}.

\subsection{Rebase Tokens}

Meanwhile, breakthroughs were made on the cryptocurrencies side as well. Rebasing --- a novel technique applied for synthetic commodities, has proved its efficiency in tackling high volatility by algorithmically rebalancing tokens supply.

\section{Protocol}

\bibliographystyle{ieeetr}
\bibliography{references}

\end{document}