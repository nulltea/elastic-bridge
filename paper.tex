\documentclass{article}

\usepackage[english]{babel}

\usepackage[a4paper, top=2cm, bottom=2cm, left=3cm, right=3cm, marginparwidth=1.75cm]{geometry}

\usepackage{amsmath}
\usepackage{graphicx}
\usepackage[colorlinks=true]{hyperref}
\usepackage{footnote}
\usepackage{enumerate}

\title{ElasticBridge: Trustless Cross-Chain Bridge for Transferring Rebase Tokens}
\author{Timothy Yalugin\\timauthx@gmail.com}

\begin{document}
\maketitle

\begin{abstract}

This paper proposes a practical approach for secure and robust bridging of ERC-20 compatible rebase tokens between EVM-based source chain and WASM-based receiver chain via the IBC interoperability protocol.

\end{abstract}

\section{Introduction}

Emerging applications of blockchain technology indicate an ongoing trend for market diversification. Once the idea of sovereignty-preserving application-specific chains became feasible, it quickly displaces a single \textit{chain-maximalism} dogma. We are now live in a world where \textit{multi-chain} networks like Polkadot \cite{wood2016polkadot} and Cosmos \cite{kwon2016cosmos} are gathering diverse interconnected applications around their relay-chains and hub-chains respectively, creating \textit{the Internet of Blockchains\footnote{Futuristic notion proposed by the \href{https://cosmos.network}{Cosmos} developers.}}.

Frameworks like Parity's Substrate\footnote{The blockchain framework
targeting Polkadot multi-chain network. See \href{https://substrate.io}{substrate.io}} and Tendermint's Cosmos SDK\footnote{Framework of development zone-blockchains for Cosmos network. See \href{https://tendermint.com/sdk}{tendermint.com/sdk}} streamlined the process of building and deploying sovereign and flexible blockchains, allowing them to have different consensus mechanisms, finalities, state models, governance, etc.

\subsection{Cross-chain Bridges}

Effectively improved on scalability by sharding on-chain data and parallelizing transaction execution, we encountered new challenges with cross-chain interoperability. Many newborn ecosystems, all with unique features and value propositions, found themselves near of Tower of Babel\footnote{Idiom that references a mythical narrative when people lose the ability to communicate with the same language. } --- unable to communicate and effectively cooperate to provide meaningful services together.

The natural reaction was the development of cross-chain bridges --- protocols that allow two (or more) exogenous chains to share state by the means of interchain transaction passing. We have seen various implementations of these protocols, ranging from trusted federations to truly trust-less decentralized Relayed SPV solutions like the IBC Protocol \cite{goes2020ibc}.

\subsection{Rebase Tokens}

Meanwhile, cryptocurrencies --- the original use case of the Distributed Ledger Technology (DLT), also do not show any signs of innovation slowdown. Projects like Stellar \cite{mazieres2016stellar} commit to building a foundation for transferring digital currency to fiat money across borders. Controversial privacy-oriented currencies like Monero allow completely untraceable payments. Finally, fiat-pegged tokens like Tether and multi-collateralized stablecoins like MakerDAO's Dai have been one of many attempts to tackle the price-volatility --- issue often synonymous with digital currencies.

Rebasing --- a novel technique applied for synthetic commodities for achieving practical price stability by algorithmically adjusting circulating tokens amount, thus translating price-volatility into supply-volatility. Rebase tokens (or price-elastic tokens) demonstrated low correlation not only with stocks, fiat currencies, and precious metals, but with other cryptocurrencies, most notably with Bitcoin.

While the low correlation aspect attracts investors seeking to diversify their portfolios, the unique volatility pattern of rebase currencies makes them a valuable building block\footnote{Claim taken from Ampleforth \href{https://ampleforth.org/economics}{documentation}.} for DeFi as well. For example, the longest-running algorithmic stablecoins based on rebasing --- Ampleforth \cite{kuo2019ampleforth} constructs its protocol with dependence on fast actors who benefit from short-term trades. By leveraging certain windows in time during rebase to make lucrative trades, these actors propagate adjustments of supply made by protocol back into the price.

\subsection{Outlined challenges}

Notably, supply expansions and contractions --- actions performed by rebase protocol to adjust total supply, affect token amount universally: all wallet balances, treasury, and accounts reserved for smart contracts. The last introduces a significant consideration for cross-chain bridges design since it generally relies on escrow account where dispatched tokens would remain locked on the source chain.

Assuming that the bridge protocol will mint a wrapped representation of the original token on the target chain, we can draw a corresponding conclusion that \textit{without changes to original bridging protocol wrapped token supply will eventually become out of sync with rebase-driven native token supply.} The inconsistency between amounts of native and wrapped tokens would then lead to non-deterministic behavior, mostly obvious --- \textit{translation conflicts when transferring rebase tokens back to the source chain.} This would also expose both chains to novel attack vectors that would not otherwise be possible.

\subsection{Proposal statement}

The purpose of this paper is to firstly shed light on the implications that elastic supply of dispatchable tokens would have on cross-chain interoperability protocols applied to them. We will then go through key considerations that should be taken into account while designing a bridge for transferring such tokens. The Paper will center around the protocol proposal for the bridge connecting EVM-compatible source chain and WASM-based chain on the Kusama network. Finally, we will finish current paper with follow-up questions and ideas for future exploration.

\section{Design consideration}

The general method by which cross-chain bridges operate can be described as \textbf{lock-and-mint/burn-and-release\footnote{Term taken from ChainBridge \href{https://chainbridge.chainsafe.io/live-evm-bridge}{documentation}.}}. Bridge would lock the tokens in the source chain (SC) and mint a wrapped representation onto the receiver chain (RC). The reverse process would then be to burn wrapped tokens from RC and unlock original tokens on SC. The described approach allows to \textit{preserve circulating token supply by distributing it across both chains}. This property will also propagate with each following transfer in N-Way bridge, where tokens can be passed downwards to another ``satellite'' chains without going back to the source chain, as long as the contract of burning wrapped tokens on leaving ``satellite'' chains will be kept.

The given overview of the conventional bridging protocol allows us to distinguish two states, where we should consider exceptions when such protocol is applied to rebase tokens. These states are 1) forward movement from source chain to target chain 2) backward movement back to source chain. In the upcoming sections we will go through each of them.

\subsection{Forward bridge movement}

Native tokens (NT) locked on the source chain escrow account would act as collateral for newly minted wrapped tokens (WT) on the receiver chain. Since WT are pegged by NT on 1:1 basis, their price ratio will initially have the same relation as well. We are now able to add new assumptions to the table:

\begin{enumerate}
	\item As long as WT are pegged with same amount of NT the price ratio will remain unchanged
	\item If after rebasing on SC there will not be an equal adjustment of NT supply, then the amount of WT become out of sync with NT so as their prices
	\item If supply adjustments made by rebase protocol will propagate to RC, then the amount of NT and WT will remain equal so as their price\footnote{Impact on price will not be instant, but rather with rebase reaction lag, that is configured by protocol or governance}
\end{enumerate}

Basing on defined assumptions, we can immediately outline first and foremost consideration that directly translates to protocol scope requirement --- \textit{in order to preserve price and supply synchrony, it is required to ensure that rebasing rules of source chain are followed on all receiver chains.}

Reasons why price-ratio and supply-ratio of wrapped and native tokens must remain equal will be covered in following sections. For now please assume that \textit{since the wrapped token is by definition a direct representation of the original token we generally want their price to be synced so as the supply.}

We also need to consider possible negative implications of locking tokens to the rebase process itself. Consider the formula for calculating supply adjustment coefficient used in Ampleforth's Supply Policy Contract \cite{kuo2019ampleforth} when current Ample\footnote{Ampleforth protocol native token} price is outside the threshold of the target price:
\[supplyDelta = \frac{currentPrice-targetPrice}{targetPrice}\ast\\totalSupply\]

Since the formula takes into account exactly the total supply, which does include the amount of tokens locked in the bridge's escrow account, we can in fact justify that dispatch of rebase tokens from their source chain is possible. The opposite would be true for circulating supply, which will decrease by the number of locked tokens.

\subsection{Backward bridge movement}

By the means of locking dispatchable tokens in an internal escrow account, bridge protocol ensures its ability to redeem the required amount of tokens back without changing total supply when minting new ones. The obvious problem with rebase currencies bridging is the fact that since the escrow account balance is a variable out of the bridge's full control, we cannot guarantee the fulfillment of the redeeming contract. We will address this issue as \textit{translation conflict} later on.

Since translation conflict is a type of non-deterministic behavior that is caused by the deviations occurring between rebase-driven native token supply and relatively static\footnote{Meaning that wrapped token supply is driven only by bridging protocol rather than rebasing} wrapped token supply, we can re-use previously defined assumptions to conclude --- \textit{similarly how the price of wrapped tokens relies on the collateral supply synchrony, the bridge protocol rely on it to fulfill its redeeming contract, and thus it must be preserved.}

\subsection{Price-Supply equilibrium}

The core process of rebase protocol is to actively seek a price-supply equilibrium --- state when current token supply constitutes into the target price through exchange-rate deviations caused by fast actors. This contract is what defines rebase currencies value to potential investors, traders, and other type of DeFi actors. Assuming so, the case with wrapped token that represents rebase currency but do not comply with its supply adjustment rules would probably make no meaningful sense for all actors besides those that are either malicious or driven by greed with similar aftermath.

We can foresee a potential attack vector derived from the native-to-wrapped token supply deviations caused in chains that lack rebasing support: \textit{a malicious fast actor can transfer to non-rebasing chain tokens, bought during expansion phases when price is lower, in order to redeem them after contraction phase when price is higher.} This gives a malicious actor a clear advantage over honest actors, since account balance on the non-rebasing chain will not contract, whereas all source chain balances will.

\subsection{Archiving supply synchrony}

Let us now define possible approaches of structuring multi-chain version of rebase protocol. These are the ones that will be outlined by the current paper:

\begin{enumerate}[a.]
	\item Rely on source chain to govern rebase calculation and then on protocol to propagate adjustments to each satellite chains
	\item Let each satellite chain govern its own wrapped token supply individually, yet with respect to source chain defined rules\footnote{Presumably, multi-chain governance of rebasing is possible, yet it will be outside the scope of this paper}
\end{enumerate}

First approach implies to having a single relay-chain that by knowing total token supply both domestically and outside could effectively rebase on it for achieving price-supply equilibrium universally by broadcasting supply delta to all satellite chains. From the perspective of the relay-chain, all satellite chains are light clients with certain account balance that must be equally rebased.

The second approach attempts to bring more decentralization to rebase governance by partitioning supply adjustments across all multi-chain participants. Although at the first glance this method does seem more advantageous, it has some caveats we have to consider first. Assuming that each individual chain will track its local total supply to perform rebase calculation, bridge by minting new tokens will introduce an additional factor that is not necessarily considered by the rebase protocol. We can of course argue that since bridge being a de facto sole-producer of such tokens on satellite chains it would also act as a rebase governor, but the described issue goes further into economical scope --- \textit{addition or reduction of tokens by means of minting or burning would most certainly alter the price-supply equilibrium established since the last rebase recurrence.} Furthermore, since we cannot deterministically assume that supply information will be equally propagated to price by each individual group of fast actors trading different wrapped token representations, there is a prominent possibility that \textit{as soon as the underlying prices diverge to the extent of target price threshold, supply deviation will most certainly happen too.}

Presumably, it still is possible to get over the described issues with some novel layer zero consensus mechanism, yet considering that use of central relay-chain for multi-chain interoperability has proved its robustness and security, the second approach will probably be an overhead.

\section{Protocol implementation}

ElasticBridge --- is a proposed implementation of cross-chain bridge connecting EVM and WASM chains for exchanging rebase tokens in a decentralized, secure and trustless way. The base chain that holds an ERC-20 token Contract is deployed on Ethermint network, that EVM-compatible, Proof-of-Stake and build with Cosmos SDK framework. On the other side of the bridge we got application-specific chain with WASM-based runtime powered by Substrate framework and deployed on Kusama\footnote{Risk-taking and fast-moving multi-chain canary network for its cousin Polkadot.} network.

Building a robust, secure, and trustless bridge requires a decentralized governance structures implemented through solid technology solutions along with a well-balanced incentive model based on an elaborate game theory. Thus, similarly to building a blockchain from scratch, bridging a few of them without any foundation is an immensely difficult task. That is why ElasticBridge will rely on existing solutions.

\subsection{Toolchain}

At the time of writing, there are two prominent candidates for EVM-to-Substrate bridge development frameworks to consider: ChainBridge\footnote{Modular highly-composable cross-chain bridge development framework developed by ChainSafe. See its \href{https://chainbridge.chainsafe.io}{docs}} and SnowBridge\footnote{Polkadot-Ethereum oriented bridge solution that facilities light clients. See its \href{https://snowbridge-docs.snowfork.com}{docs}}. Although current version of the ChainBridge operates under a trusted federation model and the SnowBridge is claimed to be trustless SPV solutions, the more likely choice out of two would still be ChainBridge, which according to ChainSafe, the developers behind the project, is a planned to move to the more trustless model as well. The developers of Ampleforth made a similar decision for their AMPL Bridge as it is powered by Meter Passport\footnote{N-Way bridge that is build with ChainBridge framework. Docs \href{https://docs.meter.io/passport}{reference}}. However, in a situation like this, rather than coping popular design decisions for a similar problem, the vision for ElasticBridge is to instead contribute to protocol implementation diversity, which provenly will make it more robust and secure.

By leveraging the fact that Ethermint is a part of Cosmos ecosystem, it is also possible to use industry-proven IBC protocol. Projects that already laid foundation for IBC support on Substrate are QuantumTunnel\footnote{Relayer that between Cosmos and Polkadot WASM interpreted light clients, \href{https://github.com/ChorusOne/quantum-tunnel}{developed} by Chorus One.}(asymmetric light client relayer) and \texttt{substrate-ibc} pallet\footnote{Substrate runtime module that brings IBC support \href{https://github.com/octopus-network/substrate-ibc}{developed} by the team behind Octopus Network} (native Rust implementation of IBC). Although both solutions are in early development stages, the fact that \texttt{substrate-ibc} pallet is basically a wrapper for Interchain\footnote{Foundation behind IBC protocol} approved \texttt{\href{https://github.com/informalsystems/ibc-rs}{ibc-rs}} library would most likely be a decisive for ElasticBridge's foundation.

\subsection{Specification}

ElasticBridge relies on IBC protocol to trust-lessly relay rebase token transfers\footnote{Outlined by \href{https://github.com/cosmos/ibc/tree/master/spec/relayer/ics-018-relayer-algorithms}{ICS18}} in the multi-chain network with unknown topology. Its own role in this kind of mutual relationship is to expose Substate-based nodes containing ElasticBridge protocol runtime. These nodes together form an intermediary blockchain acting equally as a service-providing Parachain on Kusama network and Peg Zone between the Ethermint and other Kusama or Polkadot based Parachains. Combined security guarantees of both sender chain's (Terndermint BFT) and receiver chain's (BABE + GRANDPA) consensus mechanism is what makes ElasticBridge protocol so robust and preserves its ability to provide deterministic finality for cross-chain transactions.

What differentiates ElasticBridge from similar Relayed SPV solutions, is a specific mechanism implemented in its nodes that ensure systematic, provable, and atomic supply rebasing propagation. The concrete implementation of such mechanism may vary depending on how Supply Policy Contract (SPC) is implemented on the base chain. This paper foresees the following variations:

\begin{itemize}%[leftmargin=1em]
  \renewcommand{\labelitemi}{$\Rightarrow$}
	\item In case the existence of satellite chains is recognized by SPC, and it will systematically transact a report shortly after supply delta is calculated, then ElasticBridge on-chain runtime will need to implement extrinsic for handling such reporting transaction
	\item Otherwise, all or particular ElasticBridge nodes will have an off-chain worker that will periodically scan base chain for the events indicating occurred rebasing and then determine new supply delta 
\end{itemize}

Regardless of the circumstances, ElasticBridge runtime by mimicking supply expansion and contraction of the base chain will then adjust wrapped token supply proportionally to the discovered delta coefficient. Most likely, to avoid generating a transaction for each wallet, it will only update the internal denomination that is used for representing rebase token balances. Lastly, it will propagate the same rebase transaction further to the upstream to universally synchronize supply across all participating chains.

On the practical side, programmable primitives provided by the Substrate framework is more than enough to effectively support all mentioned requirements. For example, its off-chain workers feature allows for scheduling and running non-deterministic and CPU-intensive processes that cannot otherwise be done on-chain: cross-chain transactions, RPC invocations, state indexing and more. Additionally, Substrate's fork-less\footnote{One of the defining features of the Substrate that is made possible by putting entire runtime logic on chain as WASM blob. Read more at \href{https://docs.substrate.io/tutorials/v3/forkless-upgrades/}{docs}} upgrades scheme allows rapidly adapting to any possible changes introduced by the external rebase protocol later on.

\section{Conclusion}

Rebase currencies are a promising and refreshing addition to the existing DeFi ecosystem. While the market cap of such novel synthetic commodities is driven both by their underlying price and supply signals, its proportional value to other more established cryptocurrencies is left to be seen. But the fact that prominent projects like Ampleforth are actively pushing this idea further, makes contributions to rebase token cross-chain interoperability increasingly valuable for the crypto space as a whole.

Design decisions for proposed ElasticBridge protocol are driven by the close study of previously done work and the desire to both extend platforms adoption and diversify technological implementation. In the core of current solution is IBC protocol --- blockchain-agnostic interoperability protocol that while still being mostly limited to Cosmos ecosystem, is now slowly making its way to other big platforms like Ethereum, Polkadot, and Near. The goal of ElasticBridge is to support this movement with the help of other powerful technology like Substrate and Ethermint.

\subsection{Open Questions}

\begin{itemize}
	\item Can the fact that rebase calculation would be done on a single chain introduce a single point of failure in the multi-chain setup?
	\item How can one benefit from parallelizing rebase calculation on all chains if this would be possible? Same for multi-chain governance of rebasing rules and parameters
\end{itemize}

\section{Acknowledgment}

Many thanks to Blockchain Solutions team at ChainSafe for challenging me with such an intriguing and enlightening task.

\bibliographystyle{abbrv}
\bibliography{references}

\end{document}
